\chapter{《泛函分析讲义(第二版)上》张恭庆 林源渠 --- 习题参考解答}

\xdyynotesetup{
  answer = {
    showanswer = false
  }
}

\section{度量空间}


\subsection{压缩映射原理}

\begin{exercise}
  证明:完备空间的闭子集是一个完备的子空间,而任一度量空间中的完备子空间必是闭子集.
\end{exercise}

\begin{answer}
  \begin{analysis}
    证明没什么太大的难度,核心就是度量空间中闭集的序列刻画和完备的定义
  \end{analysis}
  \begin{step}
    \item 先证明前半结论。
      任取闭子集的一个基本列,那么它也是大空间的基本列而大空间完备,那么就收敛,极限点由子集闭知也在子集内,那么这个子集就是完备的
    \item 再证明后半结论。
      任取完备子空间的一个收敛序列,那么它也是基本列,而由于完备那么它收敛的极限就在这个子空间内,由闭集的定义(《泛函分析讲义(第二版)上》张恭庆 林源渠 P2 定义1.1.4)知子空间是闭集
  \end{step}
\end{answer}


\begin{exercise}[title = {Newton 法}]
  设 $f$ 是定义在 $[a, b]$ 上的二次连续可微的实值函数, $\hat{x} \in(a, b)$ 使得 $f(\hat{x})=0, f^{\prime}(\hat{x}) \neq 0$. 求证: 存在 $\hat{x}$ 的邻域 $U(\hat{x})$, 使得 $\forall x_{0} \in U(\hat{x})$, 迭代序列
  \[
    x_{n+1}=x_{n}-\frac{f\left(x_{n}\right)}{f^{\prime}\left(x_{n}\right)} \quad(n=0,1,2, \cdots)
  \]
  是收敛的, 并且
  \[
    \lim _{n \rightarrow \infty} x_{n}=\hat{x}.
  \]
\end{exercise}

\begin{answer}[title = {《泛函分析学习指南》林源渠 P8}]
  考虑 $Tx \deltaeq x-\frac{f(x)}{f^{\prime}(x)}$, 则有
    \[
      \dv{x}(Tx) = 
        1 - \frac{\left(f^{\prime}(x)\right)^{2} - f(x) f^{\prime \prime}(x)}{\left(f^{\prime}(x)\right)^{2}} 
        = \frac{f(x) f^{\prime \prime}(x)}{\left(f^{\prime}(x)\right)^{2}} .
    \]
  因为 $f(\hat{x}) = 0, f^{\prime}(\hat{x}) \neq 0, f^{\prime \prime}(x)$ 在点 $\hat{x}$ 处连续, 所以 $\lim _{x \rightarrow \hat{x}} \frac{f(x) f^{\prime \prime}(x)}{ \left( f^{\prime}(x) \right) ^{2}} = 0$, 从而 $\exists \hat{x}$ 的邻域 $U(\hat{x})$, 使得
    \[
      \begin{gathered}
        \abs{\frac{f(x) f^{\prime \prime}(x)}{ \left( f^{\prime}(x) \right) ^{2}}} \leqslant \alpha < 1, \quad
        f^{\prime}(x) \neq 0 \quad(\forall x \in U(\hat{x})), \\
        \abs{Tx - Ty} = \abs{\frac{f(\xi) f^{\prime \prime}(\xi)}{(f^{\prime}(\xi))^{2}}} {\abs{x - y}} \leqslant \alpha \abs{x - y} \quad (\forall x, y \in U(\hat{x})) .
      \end{gathered}
    \]
  于是, 对 $\forall x_{0} \in U(\hat{x}), x_{n + 1} = T x_{n}(n = 0, 1, 2, \cdots)$ 是收敛的. 设 $x_{n} \rightarrow x \in U(\hat{x}), Tx = x \Longrightarrow f(x) = 0$, 联合
    \[
      \deduce[show-leftbrace = true, leftbrace-to-arrow = 1em]{
        $
        \begin{aligned}
          f(\hat{x})     =   0, \quad & \hat{x} \in U(\hat{x}), \\ 
          f(x)           =   0, \quad & x \in U(\hat{x}), \\ 
          f^{\prime}(x) \neq 0, \quad & \forall x \in U(\hat{x})
        \end{aligned}
        $
      }{
        $x = \hat{x}$
      }
    \]
  故有 $x_{n} \rightarrow \hat{x} (n \rightarrow \infty)$.

  \begin{remark}
    关键在求导后求极限,进而出现压缩的系数。但还有需要注意的地方,就是要保证压缩的映射除了定义域是一个完备的空间外,要保证值域要落在相同的完备空间,也就是映射 $T$ 必须是 self-map 的,
  \end{remark}
\end{answer}