\chapter{《泛函分析讲义(第二版)上》张恭庆 林源渠 --- 习题参考解答}

\xdyynotesetup{
  answer = {
    showanswer = true
  }
}

\section{度量空间}


\subsection{压缩映射原理}

\begin{exercise}
  证明:完备空间的闭子集是一个完备的子空间,而任一度量空间中的完备子空间必是闭子集.
\end{exercise}

\begin{answer}
  \begin{analysis}
    证明没什么太大的难度,核心就是度量空间中闭集的序列刻画和完备的定义
  \end{analysis}
  \begin{step}
    \item 先证明前半结论。
      任取闭子集的一个基本列,那么它也是大空间的基本列而大空间完备,那么就收敛,极限点由子集闭知也在子集内,那么这个子集就是完备的
    \item 再证明后半结论。
      任取完备子空间的一个收敛序列,那么它也是基本列,而由于完备那么它收敛的极限就在这个子空间内,由闭集的定义(《泛函分析讲义(第二版)上》张恭庆 林源渠 P2 定义1.1.4)知子空间是闭集
  \end{step}
\end{answer}


\begin{exercise}[title = {Newton 法}]
  设 $f$ 是定义在 $[a, b]$ 上的二次连续可微的实值函数, $\hat{x} \in(a, b)$ 使得 $f(\hat{x})=0, f^{\prime}(\hat{x}) \neq 0$. 求证: 存在 $\hat{x}$ 的邻域 $U(\hat{x})$, 使得 $\forall x_{0} \in U(\hat{x})$, 迭代序列
  \[
    x_{n+1}=x_{n}-\frac{f(x_{n})}{f^{\prime}(x_{n})} \quad(n=0,1,2, \cdots)
  \]
  是收敛的, 并且
  \[
    \lim _{n \rightarrow \infty} x_{n}=\hat{x}.
  \]
\end{exercise}

\begin{answer}[title = {《泛函分析学习指南》林源渠 P8}]
  考虑 $Tx \deltaeq x-\frac{f(x)}{f^{\prime}(x)}$, 则有
    \[
      \dv{x}(Tx) = 
        1 - \frac{(f^{\prime}(x))^{2} - f(x) f^{\prime \prime}(x)}{(f^{\prime}(x))^{2}} 
        = \frac{f(x) f^{\prime \prime}(x)}{(f^{\prime}(x))^{2}} .
    \]
  因为 $f(\hat{x}) = 0, f^{\prime}(\hat{x}) \neq 0, f^{\prime \prime}(x)$ 在点 $\hat{x}$ 处连续, 所以 $\lim _{x \rightarrow \hat{x}} \frac{f(x) f^{\prime \prime}(x)}{ ( f^{\prime}(x) ) ^{2}} = 0$, 从而 $\exists \hat{x}$ 的邻域 $U(\hat{x})$, 使得
    \[
      \begin{gathered}
        \abs{\frac{f(x) f^{\prime \prime}(x)}{ ( f^{\prime}(x) ) ^{2}}} \leqslant \alpha < 1, \quad
        f^{\prime}(x) \neq 0 \quad(\forall x \in U(\hat{x})), \\
        \abs{Tx - Ty} = \abs{\frac{f(\xi) f^{\prime \prime}(\xi)}{(f^{\prime}(\xi))^{2}}} {\abs{x - y}} \leqslant \alpha \abs{x - y} \quad (\forall x, y \in U(\hat{x})) .
      \end{gathered}
    \]
  于是, 对 $\forall x_{0} \in U(\hat{x}), x_{n + 1} = T x_{n}(n = 0, 1, 2, \cdots)$ 是收敛的. 设 $x_{n} \rightarrow x \in U(\hat{x}), Tx = x \Longrightarrow f(x) = 0$, 联合
    \[
      \deduce[show-leftbrace = true, leftbrace-to-arrow = 1em]{
        $
        \begin{aligned}
          f(\hat{x})     =   0, \quad & \hat{x} \in U(\hat{x}), \\ 
          f(x)           =   0, \quad & x \in U(\hat{x}), \\ 
          f^{\prime}(x) \neq 0, \quad & \forall x \in U(\hat{x})
        \end{aligned}
        $
      }{
        $x = \hat{x}$
      }
    \]
  故有 $x_{n} \rightarrow \hat{x} (n \rightarrow \infty)$.

  \begin{remark}
    关键在求导后求极限,进而出现压缩的系数。但还有需要注意的地方,就是要保证压缩的映射除了定义域是一个完备的空间外,要保证值域要落在相同的完备空间,也就是映射 $T$ 必须是 self-map 的,
  \end{remark}
\end{answer}

\begin{exercise}
  设 $(\scrX, \rho)$ 是度量空间, 映射 $T: \scrX \rightarrow \scrX$ 满足
    \[
      \rho(T x, T y) < \rho(x, y) \quad(\forall x \neq y),
    \]
  并已知 $T$ 有不动点, 求证: 此不动点是唯一的.
\end{exercise}

\begin{exercise}
  设 $T$ 是度量空间上的压缩映射, 求证: $T$ 是连续的.
\end{exercise}

\begin{exercise}
  设 $T$ 是压缩映射, 求证: $T^{n}(n \in \N)$ 也是压缩映射, 并 说明逆命题不一定成立.
\end{exercise}

\begin{exercise}
  设 $M$ 是 $ ( \R^n, \rho ) $ 中的有界闭集, 映射 $T: M \rightarrow M$ 满 足: $\rho(T x, T y) < \rho(x, y)(\forall x, y \in M, x \neq y)$. 求证: $T$ 在 $M$ 中存在 唯一的不动点.
\end{exercise}

\begin{exercise}
  对于积分方程
    \[
      x(t) - \lambda \int_{0}^{1} e^{t - s} x(s) \dd{s} = y(t)
    \]
  其中 $y(t) \in C[0,1]$ 为一给定函数, $\lambda$ 为常数, $\abs{\lambda} < 1$, 求证: 存在 唯一解 $x(t) \in C[0,1]$.
\end{exercise}



\subsection{完备化}

\begin{exercise}[title = {空间 $S$}]
  令 $S$ 为一切实 (或复) 数列
    \[
      x = (\xi_{1}, \xi_{2}, \cdots, \xi_{n}, \cdots)
    \]
  组成的集合, 在 $S$ 中定义距离为 
    \[ 
      \rho(x, y) = \sum_{k = 1}^{\infty} \frac{1}{2^{k}} \cdot \frac{\abs{\xi_{k} - \eta_{k}}}{1 + \abs{\xi_{k}-\eta_{k}}}, 
    \] 
    其中 $x = (\xi_{1}, \xi_{2}, \cdots, \xi_{k}, \cdots), y = (\eta_{1}, \eta_{2}, \cdots, \eta_{k}, \cdots) .$ 求证: $S$ 为 一个完备的度量空间.
\end{exercise}

\begin{exercise}
  在一个度量空间 $(\scrX, \rho)$ 上, 求证: 基本列是收敛列, 当且仅当其中存在一串收敛子列.
\end{exercise}

\begin{exercise}
  设 $F$ 是只有有限项不为 $0$ 的实数列全体, 在 $F$ 上引进距离
  \[
    \rho(x, y)=\sup _{k \geqslant 1} \abs{\xi_{k}-\eta_{k}}
  \]
  其中 $x = \{\xi_{k}\} \in F, y = \{\eta_{k}\} \in F$, 求证: $(F, \rho)$ 不完备, 并指出它的完备化空间.
\end{exercise}

\begin{exercise}
  求证: $[0,1]$ 上的多项式全体按距离
  \[
    \rho(p, q) = \int_{0}^{1} \abs{p(x)-q(x)} \dd{x} \quad (p, q \text { 是多项式 })
  \]
  是不完备的, 并指出它的完备化空间.
\end{exercise}

\begin{exercise}
  在完备的度量空间 $(\scrX, \rho)$ 中给定点列 $\{x_{n}\}$, 如果 $\forall \varepsilon > 0$, 存在基本列 $\{y_{n}\}$, 使得
  \[
    \rho (x_{n}, y_{n}) < \varepsilon \quad (n \in \N),
  \]
  求证: $\{x_{n}\}$ 收敛.
\end{exercise}



\subsection{列紧集}

\begin{exercise}
  在完备的度量空间中求证: 子集 $A$ 列紧的充要条件是 对 $\forall \varepsilon > 0$, 存在 $A$ 的列紧的 $\varepsilon$ 网.
\end{exercise}

\begin{exercise}
  在度量空间中求证: 紧集上的连续函数必是有界的, 并且达到它的上、下确界.
\end{exercise}

\begin{exercise}
  在度量空间中求证: 完全有界的集合是有界的, 并通过 考虑 $l^{2}$ 的子集 $E = \{e_{k}\}_{k = 1}^{\infty}$, 其中
  \[
    e_{k} = \{\underbrace{0,0, \cdots, 0,1}_{k}, 0, \cdots\},
  \]
  来说明一个集合可以是有界但不完全有界的.

\end{exercise}

\begin{exercise}
  设 $(\scrX, \rho)$ 是度量空间, $F_{1}, F_{2}$ 是它的两个紧子集, 求 证: $\exists x_{i} \in F_{i}(i = 1, 2)$, 使得 $\rho(F_{1}, F_{2}) = \rho(x_{1}, x_{2})$, 其中
    \[
      \rho(F_{1}, F_{2}) \deltaeq \inf \set{\rho(x, y)}{x \in F_{1}, y \in F_{2}} .
    \]
\end{exercise}

\begin{exercise}
  设 $M$ 是 $C[a, b]$ 中的有界集, 求证: 集合
    \[
      \set
        {F(x) = \int_{a}^{x} f(t) \dd{t}}
        {f \in M}
    \]
  是列紧集.
\end{exercise}

\begin{exercise}
  设 $E = \{\sin n t\}_{n = 1}^{\infty}$, 求证: $E$ 在 $C[0, \pi]$ 中不是列紧的.
\end{exercise}

\begin{exercise}
  求证: $S$ 空间 (定义见习题 1.2.1) 的子集 $A$ 列紧的充要条件是: 
    $\forall n \in \N, \exists C_{n} > 0$, 使得对 $\forall x = (\xi_{1}, \xi_{2}, \cdots, \xi_{n}, \cdots) \in A$, 有 $\abs{\xi_{n}} \leqslant C_{n}(n = 1, 2, \cdots)$.
\end{exercise}

\begin{exercise}
  设 $(\scrX, \rho)$ 是度量空间, $M$ 是 $\scrX$ 中的列紧集, 映射 $f: \scrX \rightarrow M$ 满足
    \[
      \rho(f(x_{1}), f(x_{2})) < \rho(x_{1}, x_{2}) \quad(\forall x_{1}, x_{2} \in \scrX, x_{1} \neq x_{2}) .
    \]
  求证: $f$ 在 $\scrX$ 中存在唯一的不动点.
\end{exercise}

\begin{exercise}
  设 $(M, \rho)$ 是一个紧度量空间, 又 $E \subset C(M), E$ 中的函数一致有界并满足下列 Hölder 条件:
    \[
      \abs{x(t_{1})-x(t_{2})} \leqslant C \rho(t_{1}, t_{2})^{\alpha} \quad(\forall x \in E, \forall t_{1}, t_{2} \in M),
    \]
  其中 $0 < \alpha \leqslant 1, C > 0$. 求证: $E$ 在 $C(M)$ 中是列紧集.
\end{exercise}