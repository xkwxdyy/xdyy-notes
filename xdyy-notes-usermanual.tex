\documentclass{xdyy-usermanual}

\xdyymanualsetup{
  info = {
    author            = {夏康玮},
    title             = { \cls{xdyy-notes} 文类},
    email             = {kangweixia_xdyy@163.com},
    date              = {2022-02-20},
    version           = {0.0.2},
    github-repository = {https://github.com/xkwxdyy/xdyy-notes},
    gitee-repository  = {https://gitee.com/xkwxdyy/xdyy-notes},
  }
}

\usepackage{xchoices}


\begin{document}
\maketitle
\tableofcontents



\section{计划}

本节主要列举想要编写的新功能。

\begin{enumerate}
  \item 勘误
    教材难免会有一些勘误的地方,可以汇总起来方便查找和分享。需要涉及到的参数:
      \begin{enumerate}
        \item 勘误的是什么书还是文献
        \item 勘误的版本?或者出版时间之类的补充细节
        \item 勘误的原文
        \item 勘误后的修改文
        \item 为什么是需要勘误的以及自己的理解
      \end{enumerate}
\end{enumerate}


\section{文类简介}

本文类主要用于数学读书笔记:

\begin{enumerate}
  \item 用于补充书中的细节,写读书笔记
  \item 比手写和照片更方便保存和分享
\end{enumerate}


\section{命令环境说明}


\subsection{命令}

\begin{function}[added = 2022-02-17]{\xdyynotesetup}
  \begin{syntax}
    |\xdyynotesetup| \marg{\kvopt{key}{val}}
  \end{syntax}
  全局设置,下面通过示例介绍键值:
  \begin{latexcode}
    \xdyynotesetup{
      % 笔记信息
      info = {
        author = {夏康玮},
        title = {算子理论读书笔记},
        date = {2022年2月17日}
      },
      % 前言的落款签名
      signature = {
        name = {夏康玮},
        place = {珞珈山},
        date = {2022年2月17日}
      },
      style = {
        % bib文件名
        bib-resource = {xdyy-notes-template.bib}
      }
    }
  \end{latexcode}
\end{function}


\subsubsection{ 落款命令 \tn{signature} }

\begin{function}[added = 2022-02-17]{\signature}
  \begin{syntax}
    |\signature| \oarg{name = ..., place = ..., date = ...}
  \end{syntax}
  用于摘要的落款签名,可选参数用于修改姓名,地点和日期。注意,此命令已嵌入 \env{preface} 环境中,只需要通过 \env{preface} 环境的可选参数进行修改即可。
\end{function}


\subsubsection{ 勘误命令 \tn{correction} }

一些书或文章中难免会有一些笔误甚至是错误的地方,这个时候可以使用 \tn{correction} 命令。

\begin{latexcode}
    \correction[
      book = {《数学讲义》},   % 书籍
      page = {88},             % 页数
      author = {夏康玮},       % 作者
      edition = {第一版},      % 版本
      year = {2022},           % 年份
      original = {             % 勘误前的原文
        $1+1 = 3$
      },
      revision = {             % 勘误后
        $1+1 = 2$
      },
      explanation = {          % 补充解释
        常识性问题,无须多言
      }
    ]
\end{latexcode}


\begin{latexcode}
    \correction[
      paper = {How to learn math},   % 文章
      page = {2},              % 页数
      author = {夏康玮},       % 作者
      edition = {第一版},      % 版本
      year = {2022},           % 年份
      original = {             % 勘误前的原文
        $1+1 = 3$
      },
      revision = {             % 勘误后
        $1+1 = 2$
      },
      explanation = {          % 补充解释
        常识性问题,无须多言
      }
    ]
\end{latexcode}


\subsection{环境}

\subsubsection{ 前言环境 \env{preface} }
\begin{function}[added = 2022-02-17]{preface}
  \begin{syntax}
    |\begin{preface}[signature = {name = ..., place = ..., date = ...}]|
      |  ...|
    |\end{preface}|
  \end{syntax}
  前言环境,可选参数继承于 \tn{signature} 的参数,设置落款的姓名,地点和日期
\end{function}



\subsubsection{ 引用环境 \env{quotation} }

为了一些方便可能需要在笔记中引用别的书籍文章中的原文,这个时候就需要用 \env{quotation} 环境

\begin{vexample}
    \begin{quotation}[
      book    = {《泛函分析讲义》},      % 书籍
      edition = {第二版},                % 版本
      year    = {2017},                  % 年份
      author  = {xdyy},                  % 作者
      page    = {11-14}                  % 页码
    ]\label{quo:test1}
      $E = m c^2$, 这 \ref{quo:test1} 是一个引用
    \end{quotation}
\end{vexample}

\begin{vexample}
    \begin{quotation}[
      paper   = {Function analysis notes},   % 论文、文章
      edition = {3rd},                       % 版本
      year    = {2022},                      % 年份
      author  = {xdyy},                      % 作者
      page    = {11-14}                      % 页码
    ]\label{quo:test2}
      $E = m c^2$, 这 \ref{quo:test2} 是一个引用
    \end{quotation}
\end{vexample}



\subsubsection{ 细节补充环境 \env{detail} }

读书笔记模版中最重要的部分,也就是对于书或文章内容的理解和感悟,或是证明细节的补充等等。


\begin{latexcode}
    \begin{detail}[
      paper   = {Function analysis notes},   % 论文、文章
      page = {103},                          % 页码
      author = {许全华},                     % 作者,可省略
      edition = {第一版},                    % 版本,可省略
      year = {2017},                         % 年份,可省略
      original = {                           % 原文
        ...
      }
    ]
      这段文字是用来测试效果的,没有实际含义
    \end{detail}
\end{latexcode}

\begin{latexcode}
    \begin{detail}[
      book = {《泛函分析讲义》},  % 书籍
      page = {103},               % 页码
      author = {许全华},          % 作者,可省略
      edition = {第一版},         % 版本,可省略
      year = {2017},              % 年份,可省略
      original = {                % 原文
        ...
      }
    ]
      这段文字是用来测试效果的,没有实际含义
    \end{detail}
\end{latexcode}



\subsubsection{ 习题环境 \env{exercise} }

如果老师上课布置了习题或是习题课上的内容,可以使用 \env{exercise} 环境,但是要注意,为了更好地实现模块化效果,一个 \env{exercise} 环境是一道题,该环境主要参考 zepinglee 的 \cls{exam-zh} 项目 \footnote{ \url{https://gitee.com/zepinglee/exam-zh}}。



\subsubsection{ 充分性、必要性环境 \env{sufficiency} \env{necessity}}

为了更好地实现模块化,且在证明过程中充分性和必要性更加清楚,充分性和必要性也写了一个环境。
\end{document}
